\documentclass{article}
\usepackage[T1]{fontenc}
\usepackage{fullpage}
\usepackage{enumitem}
\setlistdepth{9}

\setlist[itemize,1]{label=$\bullet$}
\setlist[itemize,2]{label=$\bullet$}
\setlist[itemize,3]{label=$\bullet$}
\setlist[itemize,4]{label=$\bullet$}
\setlist[itemize,5]{label=$\bullet$}
\setlist[itemize,6]{label=$\bullet$}
\setlist[itemize,7]{label=$\bullet$}
\setlist[itemize,8]{label=$\bullet$}
\setlist[itemize,9]{label=$\bullet$}

\renewlist{itemize}{itemize}{9}

\title{G4-STORK Input Manual}
\author{Andrew Tan \& Wesley Ford}
\date{October 2015}

\begin{document}

\maketitle

\section{Installing G4-STORK}
\begin{itemize}
\item Download tarballs \ for CLHEP-2.1.3.1, marshalgen-1.0, topc-2.5.2, geant\texttt{4.10.00.p02}, and G4STORK
\item Installing \ CLHEP-2.1.3.1

\begin{itemize}
\item \texttt{Note: this is no longer nessecary, the later verisons of geant4 come with their own version of CLHEP}
\item \texttt{http://proj-clhep.web.cern.ch/proj-clhep/DISTRIBUTION/}
\item tar -xzf clhep-2.1.3.1.tgz
\item cd 2.1.3.1
\item mkdir clhep-build clhep-install
\item cd clhep-build
\item \texttt{cmake -DCMAKE\_INSTALL\_PREFIX= [clhep-install full path] [2.1.3.1/CLHEP/ full path]}
\item \texttt{make}
\item \texttt{make test}
\item \texttt{make install}
\end{itemize}
\item \texttt{Installing marshalgen-1.0}

\begin{itemize}
\item \texttt{http://www.ccs.neu.edu/home/gene/marshalgen.html}
\item \texttt{tar -xzf marshalgen.tar.gz}
\item \texttt{yum install bison flex}
\item \texttt{cd marshalgen-1.0/}
\item \texttt{make}
\item \texttt{fix  marshalgen-1.0/Phase1/mgen.pl:683}
\begin{itemize}
\item \begin{verbatim}
from this $annot.= ”\t\t\tcopy  off \+\= sizeof\($strElementType\)\);\n”;
\end{verbatim}
\item \begin{verbatim}
to this $annot.= ”\t\t\tcopy  off \+\= sizeof\($strElementType\);\n”;
\end{verbatim}
\end{itemize}
\end{itemize}
\item \texttt{Installing \ topc-2.5.2}

\begin{itemize}
\item \texttt{http://www.ccs.neu.edu/home/gene/topc.html}
\item \texttt{tar -xzf topc-2.5.2.tar.gz}
\item \texttt{cd topc-2.5.2}
\item \texttt{./configure (installs to usr/local/include and usr/local/src}
\item \texttt{make install this may give an error but still succeed use the next step to check}
\item \texttt{make test}
\end{itemize}
\item \texttt{Installing Visualization Drivers}

\begin{itemize}
\item X11 OpenGL Visualization (\emph{Linux and Mac OS X}) 

\begin{itemize}
\item very responsive photorealistic graphics
\item \emph{Requires}: X11 headers and libraries,OpenGL or MesaGL headers and libraries. 

\begin{itemize}
\item yum groupinstall {\textquotedbl}X Software Development{\textquotedbl}
\item \emph{yum groupinstall ``Development Tools''}
\end{itemize}
\item Optional: Motif to add gui control
\end{itemize}
\item Qt User Interface and Visualization (\emph{All Platforms}) 

\begin{itemize}
\item very responsive photorealistic graphics plus more interactivity
\item \emph{Requires}: Qt4 headers and libraries, OpenGL or MesaGL headers and libraries

\begin{itemize}
\item yum install qt4-devel
\item yum groupinstall {\textquotedbl}X Software Development{\textquotedbl}
\item \emph{yum groupinstall ``Development Tools''}
\end{itemize}
\end{itemize}
\item \emph{RayTracer Visualization (Linux and Mac OS X) }

\begin{itemize}
\item \texttt{install ghostscript}

\begin{itemize}
\item yum install ghostscript
\end{itemize}
\item install ghostview

\begin{itemize}
\item http://pages.cs.wisc.edu/\~{}ghost/gsview/get49.htm
\item or http://www.filewatcher.com/m/gsview-4.9.tar.gz.892681-0.html
\item \emph{setenv DAWN\_PS\_PREVIEWER {\textquotedbl}ghostview{\textquotedbl}}
\end{itemize}
\end{itemize}
\item X11 RayTracer Visualization (\emph{Linux and Mac OS X}) 

\begin{itemize}
\item visualize a geometry that the other visualization drivers can't handle
\item \emph{Requires}: X11 headers and libraries. 

\begin{itemize}
\item yum groupinstall {\textquotedbl}X Software Development{\textquotedbl}
\end{itemize}
\end{itemize}
\item \texttt{DAWN}

\begin{itemize}
\item highest quality photorealistic images\texttt{ with little control}
\item \texttt{install }Tcl/Tk 

\begin{itemize}
\item \texttt{maybe preinstalled, try wish command to see}
\item \texttt{if not install from http://www.tcl.tk/software/tcltk/8.6.html}
\end{itemize}
\item \texttt{install ghostscript}

\begin{itemize}
\item yum install ghostscript
\end{itemize}
\item install ghostview

\begin{itemize}
\item http://pages.cs.wisc.edu/\~{}ghost/gsview/get49.htm
\item or http://www.filewatcher.com/m/gsview-4.9.tar.gz.892681-0.html
\item setenv DAWN\_PS\_PREVIEWER {\textquotedbl}ghostview{\textquotedbl}
\end{itemize}
\item Install DAWN

\begin{itemize}
\item http://geant4.kek.jp/\~{}tanaka/DAWN/About\_DAWN.html
\item tar -xzf dawn\_version.tgz
\item cd dawn\_version
\item make clean
\item make guiclean
\item ./configure
\item make
\item make install
\end{itemize}
\item Optional install david to see intersecting geometry

\begin{itemize}
\item http://geant4.kek.jp/\~{}tanaka/DAWN/About\_DAVID.html
\item \begin{verbatim}
tar -xzf david_1_36a.taz
\end{verbatim}
\item \begin{verbatim}
cd  david_1_36a
\end{verbatim}
\item \begin{verbatim}
make -f Makefile.GNU_g++
\end{verbatim}
\item \begin{verbatim}
ls -F david
\end{verbatim}
\begin{verbatim}
(Please check if a binary file named "david" is created.)
\end{verbatim}
\item \begin{verbatim}
Optional: Test to make sure it installed correctly
\end{verbatim}
\begin{verbatim}
cd TESTDATA
\end{verbatim}
\begin{verbatim}
dawn -c
\end{verbatim}
\begin{verbatim}
export DAVID_DAWN_PVNAME=1
\end{verbatim}
\begin{verbatim}
../david david_boxes.prim 
\end{verbatim}
\begin{itemize}
\item Go to page 1 of the DAWN's GUI panel.
\item Click the "Load Default" button
\item select "Surface" for the "viewing mode"\\
\item Display Parameters and Axes to Yes\\
\item Go to page 4 of the DAWN's GUI panel, and select "EPS" in the menu of the "Device Selection"
\item Click the "Save Default" button
\item Click the "Okay" button
\item Confirm the followings: Two of three boxes, named Box1 and Box2, are highlighted with "red color" and with forced wireframe style. The remaining one box is visualized with gray and with surfaces.
\end{itemize}
\item make install
\end{itemize}
\end{itemize}
\end{itemize}
\item \texttt{Installing geant4.10.00.p02}

\begin{itemize}
\item yum install expat-devel 
\item \texttt{tar -xzf geant4.10.00.p02.tar.gz}
\item \texttt{mkdir geant4.10.00.p02{}-build geant4.10.00.p02{}-install}
\item \texttt{cd \ geant4.10.00.p02{}-build}
\item \texttt{cmake -DGEANT4\_INSTALL\_DATA=ON -[Visualization Drivers]\\ -DCMAKE\_INSTALL\_PREFIX=/path/to/geant4.10.00.p02{}-install /path/to/geant4.10.00.p02}
\item \texttt{make -j[\# of processors]}
\item \texttt{make install}
\item \texttt{Note: to add data manually instead of automatically uploading with make}

\begin{itemize}
\item \texttt{mkdir /path to geant4.10.00.p02-install/share/Geant4-10.0.2/data/}
\item \texttt{cp G4ABLA.3.0.tar.gz G4EMLOW.6.35.tar.gz G4ENSDFSTATE.1.0.tar.gz G4NDL.4.4.tar.gz G4NEUTRONXS.1.4.tar.gz G4PhotonEvaporation.3.0.tar.gz G4PII.1.3.tar.gz \\G4RadioactiveDecay.4.0.tar.gz G4SAIDDATA.1.1.tar.gz RealSurface.1.0.tar.gz /path to geant4.10.00.p02-install/share/Geant4-10.0.2/data/}
\item \texttt{cd /path to geant4.10.00.p02-install/share/Geant4-10.0.2/data/}
\item \texttt{tar -xzf [data]}
\item \texttt{source \$\{GEANT4 INSTALL\}/bin/geant4.sh}
\end{itemize}
\item \texttt{Note: to add data for Z{\textgreater}92}

\begin{itemize}
\item \texttt{contact cern and retrieve added data}
\item \texttt{mv extraElemData.tar.gz /path/to/geant4.10.00.p02{}-install/share/Geant4-10.0.2/data/}
\item \texttt{cd /path/to/geant4.10.00.p02{}-install/share/Geant4-10.0.2/data/}
\item \texttt{mkdir ExtraData}
\item \texttt{tar -xzf \ extraElemData.tar.gz -C ExtraData/}
\item \texttt{cp -nr ExtraData/* G4NDL4.4/}
\item \texttt{rm -rf ExtraData/ \ extraElemData.tar.gz}
\end{itemize}
\end{itemize}
\item \texttt{Installing G4STORK}

\begin{itemize}
\item \texttt{tar -xzf \$\{G4-STORK TARBALL\} \$\{DESTINATION\}}
\item \texttt{cd \ G4STORK/Build/}
\item \texttt{source \$\{GEANT4 INSTALL\}/bin/geant4.sh}
\item \texttt{rm -rf CMakeCache.txt CMakeFiles/ }
\item \texttt{cmake -DTOPC\_USE=1 -[Vis and debug options] -DGeant4\_DIR=\$\{GEANT4 INSTALL\}/lib64/Geant4.10.00.p02/Geant4Config.cmake ../}

\begin{itemize}
\item Options:

\begin{itemize}
\item \texttt{{}-DCMAKE\_BUILD\_TYPE=Debug,[Release] controls whether debuggin information is added to the code}
\item \texttt{{}... the usual compiler options}
\item \texttt{{}-DG4TIME\_USE=[0],1 controls whether timing information the efficiency of the program is outputted}
\item \texttt{{}-DG4VERBOSE\_TRACKING\_USE=[0],1 controls the amount of neutron tracking information that is outputted }
\item \texttt{{}-DG4VISUALIZE\_USE=[0],1 controls whether the GEANT4 user interface pops up for visualizing the given geometry}
\item \texttt{{}-DSTORK\_EXPLICIT\_LOSS=[0],1 controls whether the number of neutrons lost to each process is printed for each event }
\item \texttt{{}-DTOPC\_USE=[0],1 controls whether the code is run in parallel using an MPI frame work}
\item \texttt{{}-DTOPC\_USE\_SEQ=[0], 1 forces parallel code to run like sequential code for debugging purposes }
\item \texttt{{}-DWITH\_GEANT4\_UIVIS=[1],0 allows for user interface to appear }
\end{itemize}
\end{itemize}
\item \texttt{make -j[\# of processors]}
\end{itemize}
\end{itemize}

\section{Input Options}
\subsection{Required Inputs}
\subsubsection{Simulation World}

This option determines which simulation world is to be used, the current options are listed in the table below:

\begin{verbatim}
WORLD <OPTION>
\end{verbatim}

\begin{figure}[h!]
\centering
\begin{tabular}{|c|c|}
\hline
\verb OPTION & Description \\
\hline
\verb Sphere & A sphere of U235\\
\verb C6Lattice & Lattice cell of the CANDU6\\
\verb Cube & \\
\verb ZED2 & The full core ZED-2 reactor\\
\verb Q_ZED2 & The quarter core ZED-2\\
\verb SLOWPOKE & The SLOWPOKE-2 reactor (RMC)\\
\verb SCWR & The (lattice cell?) Super-Critical Water Reactor.\\
\verb SCWRDoppler & \\
\verb SCWRJason & \\
\hline
\end{tabular}
\caption{World options}
\label{fig:my_label}
\end{figure}

\subsubsection{Events, Primaries and Duration}
These inputs control the simulation scheme.\\
\begin{verbatim}
INPUT <integer>
\end{verbatim}
\begin{figure}[h!]
\centering
\begin{tabular}{|c|c|}
\hline
\verb INPUT & Description \\
\hline
\verb NUM_RUNS  & Sets the number of runs in the simulation. \\
\verb NUM_EVENTS & Sets the number of events per run.\\
\verb NUM_PRIMARIES_PER_EVENT & Sets the number of primaries per event/\\
\verb RUN_DURATION & Sets the run duration in nanoseconds.\\
\hline
\end{tabular}
\caption{Simulation inputs.}
\label{fig:my_label}
\end{figure}
\newpage
\subsection{Additional Inputs}
\subsubsection{Shannon Entropy}
\subsubsection{Initial Particles}
\subsubsection{Logging Files}


%%%%% Added by Andrew Tan, 2015-Dec-16 %%%%%%
\subsection{Delayed Neutron Options}
This section outlines the 4 main ways we simulate delayed neutrons in G4-STORK. The \texttt{DELAYED\_OPTION} input has the following options:
\begin{description}
    \item[No delayed neutrons] \hfill \\
    \texttt{DELAYED\_OPTION 0} -- Delayed neutrons are not simulated.
    \item[Instantaneous delayed neutrons] \hfill \\
    \texttt{DELAYED\_OPTION 1} -- Delayed neutrons are set to appear the moment of fission.
    \item[Natural Creation] \hfill \\
    \texttt{DELAYED\_OPTION 2} -- Delayed neutrons are created at the the time of fission and are set to appear sometime in the future.
    \item[Precursor-Decay Method] \hfill \\
    \texttt{DELAYED\_OPTION 3} -- An initial fission source file is used to generate a precursor distribution (six groups) and the precursors decay naturally into the simulation. Precursors are created from fission.
    
\end{description}

%%%%% Added by Andrew Tan, 2015-Dec-16 %%%%%%
\subsection{SLOWPOKE heat transfer}
This section outlines the heat transfer specifically pertaining to the SLOWPOKE-2 reactor. For more information refer to the M.A.Sc thesis \textit{TRANSIENT SIMULATIONS OF THE SLOWPOKE-2 REACTOR USING G4-STORK}.
\begin{description}
    \item[\texttt{RUN\_THERMAL\_MODEL}] \hfill \\
    Takes inputs 0 or 1. Turns the heat transfer model on or off.
    \item[\texttt{SET\_HEAT\_TRANSFER\_COEFFICIENT}] \hfill \\
    Input for the heat transfer coefficient between the fuel and the coolant. In units of (W m$^{-2}$ K$^{-1}$).
    \item[\texttt{SET\_AMBIENT\_TEMPERATURE}] \hfill \\
    Sets the coolant (ambient) temperature to which the heat will dissipate into. In units of K.
    \item[\texttt{SET\_BASELINE\_FISSION\_RATE}] \hfill \\
    Set the baseline rate of fissions. This number should be calculated or inferred from a previous set-up simulation. Units are in fissions/ns.
    \item[\texttt{SET\_FISSION\_TO\_ENERGY\_COEFFICIENT}] \hfill \\
    Sets the amount of energy a fission produces. Units are in Joules.
\end{description}

%%%%% Added by Andrew Tan, 2015-Dec-16 %%%%%%
\subsection{Calculation types}
This section outlines the input \texttt{CALC\_TYPE} and its various options. It should be noted that when doing transients the precursor-delayed option should be used which means a matching source and fission source file are needed.
\begin{description}
    \item[STATIC RENORMALIZATION:] \hfill \\
    \texttt{CALC\_TYPE 0} -- Static renormalization samples primaries from a given source file for each run. This should be utilized for static calculations where the input source is converged.
    \item[DYNAMIC NONRENORMALIZATION:] \hfill \\
    \texttt{CALC\_TYPE 1} -- From an input source the primaries will be renormalized  \textit{only at the start} to the set number of primaries. Then the simulation will continue with no-renormalization used.
    \item[TRANSIENT CONTINUATION:] \hfill \\
    \texttt{CALC\_TYPE 2} -- From an input source primaries will be loaded and the simulation will continue with no renomalization used. This simulation type allows the user to directly continue from a previous simulation. Useful for transient simulations.
    
\end{description}



\end{document}
